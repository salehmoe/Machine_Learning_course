




\documentclass[journal]{IEEEtran}
%
% If IEEEtran.cls has not been installed into the LaTeX system files,
% manually specify the path to it like:
% \documentclass[journal]{../sty/IEEEtran}
\usepackage{graphicx}
\usepackage{listings}
\lstset{frame=tb,
  language=Python,
  basicstyle=\fontsize{6}{6}\selectfont\ttfamily,
%  aboveskip=3mm,
%  belowskip=3mm,
%  showstringspaces=false,
%  columns=flexible,
%  basicstyle={\small\ttfamily},
%  numbers=none,
%  numberstyle=\tiny\color{gray},
%  keywordstyle=\color{blue},
%  commentstyle=\color{dkgreen},
%  stringstyle=\color{mauve},
 breaklines=true,
breakatwhitespace=true
%  tabsize=3
}




% *** GRAPHICS RELATED PACKAGES ***
%
\ifCLASSINFOpdf
  % \usepackage[pdftex]{graphicx}
  % declare the path(s) where your graphic files are
  % \graphicspath{{../pdf/}{../jpeg/}}
  % and their extensions so you won't have to specify these with
  % every instance of \includegraphics
  % \DeclareGraphicsExtensions{.pdf,.jpeg,.png}
\else
  % or other class option (dvipsone, dvipdf, if not using dvips). graphicx
  % will default to the driver specified in the system graphics.cfg if no
  % driver is specified.
  % \usepackage[dvips]{graphicx}
  % declare the path(s) where your graphic files are
  % \graphicspath{{../eps/}}
  % and their extensions so you won't have to specify these with
  % every instance of \includegraphics
  % \DeclareGraphicsExtensions{.eps}
\fi


% correct bad hyphenation here
\hyphenation{op-tical net-works semi-conduc-tor}


\begin{document}
%
% paper title
% Titles are generally capitalized except for words such as a, an, and, as,
% at, but, by, for, in, nor, of, on, or, the, to and up, which are usually
% not capitalized unless they are the first or last word of the title.
% Linebreaks \\ can be used within to get better formatting as desired.
% Do not put math or special symbols in the title.
\title{Airplane crashes between Numbers and Regulations}
%
%
% author names and IEEE memberships
% note positions of commas and nonbreaking spaces ( ~ ) LaTeX will not break
% a structure at a ~ so this keeps an author's name from being broken across
% two lines.
% use \thanks{} to gain access to the first footnote area
% a separate \thanks must be used for each paragraph as LaTeX2e's \thanks
% was not built to handle multiple paragraphs
%

\author{Mohammad Saleh}



% make the title area
\maketitle

% As a general rule, do not put math, special symbols or citations
% in the abstract or keywords.
\begin{abstract}
Airplane accidents, still the most dangerous life threatening accident. The main reason for this believe or fearness is the
number of survivor across each accident. Nowadays, it is a fact that nobody will survive an airplane accident. The global aviation
administrations, mainly FAA (Federal aviation administration) in USA, regularly release new rules to maintain and decrease the number of airplane
crashes. Many people show the statistics for these crashes but the results shows that this number is slightly decreasing over the years
because they didn't take into account the number of flights yearly. In this letter, we will present the latest analysis for the number
 of crashes per year globally and among the nations taking into account the increasing number in the flights yearly. We will highlight
 the main causes of the crashes and the proposed solutions for the fulfillment of the final project of the machine learning course.
\end{abstract}

% Note that keywords are not normally used for peerreview papers.
\begin{IEEEkeywords}
crashes, paper, analysis, airplane.
\end{IEEEkeywords}

\IEEEpeerreviewmaketitle



\section{Introduction}
\IEEEPARstart{A}{irplane} is one the greatest human inventions. Nowadays, the technology for the new inventions in the
aerospace field is growing fast, mainly in the drones technology. Lately, the FAA in the USA released a new regulations regarding
 drone operations to maintain the safety
of the commercial airplanes. In this analysis, we wanted to check how effective the regulations and safety on the airplane crashes
and the number of survivors. It is very
well known that the FAA is one of the main administrations for airplanes safety. We will start with general overview of the crashes
yearly and the main reasons and check the proportionally of airplane crashes number in the USA.
% You must have at least 2 lines in the paragraph with the drop letter
% (should never be an issue)
%\hfill April 24, 2017
\section{Analysis}
\subsection{Crashes per year}
Many people have worked on this analysis in terms of showing the number of crashes yearly without taking into
account the number of flights yearly. This analysis has been forked from previous one at the Kaggle website. Figure~\ref{yearly} shows the number of crashes yearly normalized by the number
of passengers yearly  as we don't have direct access to the number of flights yearly. The results show the number of crashes
till 2009 because of the limitation of the dataset. The numbers on the y-axis are up to
a constant value, the most important thing is the general trend in the number of crashes. It shows that the number of
crashes decrease as times go, except couple years (1988, 1991) that had some reasons for having large number of crashes.
 In the following paragraphs, we will
look in the causes of the crashes and the main reason for this decrease in the number of crashes.
\begin{figure}
  \centering
    \includegraphics[width=0.5\textwidth]{Yearly.png}
    \caption{Proportion of the number of crashes of airplanes (commercial and non-commercial) yearly from 1996 to 2004}\label{yearly}
\end{figure}
\subsection{Cause of crashes}
Figure~\ref{cause} shows the main reason for the crashes where the y-axis is in log-scale for comparison. The highest
contribution is coming form the engine failure then bad
weather and many different reasons shown in the figure. Principal component analysis (PCA) using the first two features scatter plot
for the six different main causes shows that the engine failure is slightly correlated with other causes which is not surprising as
engine failure is mostly from the engine mechanics. Engine failure is also not correlated with the FAA regulations and safety.
To investigate more in the matter of the main cause of airplane causes, we looked for the number of crashes across the nations without
normalizing these number of crashes. This investigation will show us the effect of FAA regulations on the number of crashes
in the USA yearly.

\subsection{Crashes in USA}
Figure~\ref{nation} shows the number for crashes yearly across different nations. The y-axis plotted in log-scale for
comparison. It shows that the USA has the largest number of
accidents followed by Russia, whereas the USA is one of the leaders in the aviation regulations and safety and the third world
country doesn't show much contribution in the figure. The large number of accidents in
the USA is directly correlated to the number of flights. It is clear that this large number is directly related to the number of
flights and engine failures as the cause of the accidents. Moreover, the crashes are mainly from single-engine airplanes which
doesn't have emergency plans for engine failure. Even for multi-engine airplanes, the emergency plans for passengers still not
effective these days, but this raise more wide research in terms of what new designs of airplanes that can lead to larger number of survivors
incase of airplane crashes, many proposal are done but they still far from reality, the difficulty in this subject is due to the high altitude of the airplane.




\begin{figure}
  \centering
    \includegraphics[width=0.5\textwidth]{cause.png}
    \caption{THe main causes of airplane crashes}\label{cause}
\end{figure}
\begin{figure}
  \centering
    \includegraphics[width=0.5\textwidth]{Nations.png}
    \caption{The main causes of airplane crashes}\label{nation}
\end{figure}


\section{Conclusion}
We have shown the numbers of accidents and the main causes. The number of accidents are mainly from
technology side (engine). The main cause lead to very small number of survivor. The best approach for increasing the
number of survivors is also not effective due to the safety followed these days. To decrease the number of crashes and increase the
number of survivor, the rules should be
regarding airplanes manufacturing and emergency plans that can be followed by passengers other than the current ones. Lastly, to take you far from airplane fearness, the probability to die from
car accident is much smaller than the probability to die from airplane accident.

\newpage
\appendices
\section{Code}
\lstinputlisting{Code.py}

\begin{thebibliography}{1}

\bibitem{Kaggle}
https://www.kaggle.com/saurograndi/airplane-crashes-since-1908.
\bibitem{norm}
http://data.worldbank.org/indicator/IS.AIR.PSGR?end=2015

\end{thebibliography}

\end{document}
